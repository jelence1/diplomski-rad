\chapter{Uvod}

Internet stvari \engl{Internet of Things - IoT} brzo je rastuća tehnologija koja urbani svijet pretvara u potpuno mrežno povezani sustav visoke tehnologije. Uz sve veću popularnost, potražnju i rastuće zahtjeve na IoT tehnologiju, sustavi generiraju sve veću količinu podataka koju je gotovo nemoguće obraditi lokalno. Tom je izazovu doskočila tehnologija računarstva u oblaku \engl{cloud computing} koja pruža softver, alate i infrastrukturu preko interneta umjesto lokalnog upravljanja. Računanje u oblaku može pružiti skalabilnost i dostupnost potrebnu za obradu i pohranu velike količine podataka koju stvaraju IoT sustavi. Osim toga, računanje u oblaku može pomoći smanjiti troškove upravljanja IoT sustavima. 

U posljednjih nekoliko godina, poljoprivreda je doživjela značajne promjene zahvaljujući integraciji naprednih tehnologija poput interneta stvari i računarstva u oblaku. Ove tehnologije omogućuju razvoj moderne poljoprivrede koja koristi različite senzore i podatke radi optimizacije proizvodnih procesa i povećanja efikasnosti. IoT u poljoprivredi omogućuje povezivanje fizičkih uređaja i senzora s internetom, čime se prikupljaju podaci o stanju na poljoprivrednim gospodarstvima, stanju usjeva i učinkovitosti rada. Računarstvo u oblaku pruža infrastrukturu za obradu i analizu tih podataka u stvarnom vremenu, što omogućuje poljoprivrednicima da donose informirane odluke i poboljšavaju svoj tijek rada. 

Za razvoj poljoprivrednih IoT sustava potrebni su ugradbeni računalni sustavi s mogućnošću bežičnog povezivanja, primarno Wi-Fi komunikacije. Serija mikrokontrolera ESP32 tvrtke \textit{Espressif} jedan je od danas najpopularnijih izbora među bežičnim uređajima zbog niske potrošnje, visoke otpornosti na temperature te najvažnije, jednostavnom bežičnom povezivosti \cite{top_iot_boards}. Jedan takav integrirani sklop jest ESP32-C3 koji pruža Bluetooth i Wi-Fi povezivanje. Sklop je integriran u nekoliko modula, koji su pak dio razvojnih sustava koje proizvodi tvrtka \textit{Espressif}. Za izradu ovog rada odabran je razvojni sustav ESP32-C3-DevKitM-1. Isto tako, usluga računarstva u oblaku mora biti jednostavna, intuitivna, pouzdana te najvažnije, skalabilna. Platforma AWS tvrtke \textit{Amazon} ističe se kao jedan od najkorištenijih sustava za računarstvo u oblaku. Također nudi brojne mogućnosti za umrežavanje IoT sustava. Isto tako, potrebna je i prikladna web aplikacija koja će pružiti kvalitetne vizualizacije prikupljenih podataka. Jedna takva interaktivna aplikacija jest Grafana, koja nudi široku paletu grafikona i sučelja za prikaz podataka. 

Rad je podijeljen u cjeline kako slijedi. U drugom poglavlju \textit{„Opis sustava i tehničkih zahtjeva“} objašnjen je koncept precizne poljoprivrede kao domene izrađenog sustava te je prikazana sklopovska i programska arhitektura rješenja, kao i zahtjevi koje je trebalo ispuniti. Treće poglavlje \textit{„Razvojni sustav ESP32-C3-DevKitM-1“} opisuje osnovne značajke korištenog razvojnog sustava kao ciljane hardverske platforme, a zatim su opisana svojstva Wi-Fi mreže te njene značajke koje podržava razvojni sustav. U četvrtom poglavlju \textit{„Amazon Web Services“} opisana je korištena platforma za računarstvo u oblaku i navedene su usluge koje nudi za razvoj IoT sustava. U petom poglavlju \textit{„Povezivanje razvojnog sustava i oblaka“} opisana je programska potpora za mikrokontroler i periferne uređaje te dio programske potpore za računarstvo u oblaku koja služi za povezivanje i komunikaciju s ugradbenim sustavom. Također je opisana infrastruktura potrebna za pohranu podataka na platformi. U šestom poglavlju \textit{„Web aplikacija“} opisana je infrastruktura razvijene web aplikacije na temelju platforme Grafana. Opisan je i alarmni sustav integriran u web aplikaciju.

\eject