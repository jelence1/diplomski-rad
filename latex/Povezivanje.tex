\chapter{Povezivanje razvojnog sustava i oblaka}

Oblak koju pruža platforma AWS i razvojni sustav ESP32-C3 dva su odvojena sustava koja moraju međusobno komunicirati i razmjenjivati podatke. Za ostvarenje njihove veze razvijena su dva programska rješenja:
\begin{enumerate}
	\item programska potpora za mikrokontroler, koja će omogućiti...,
	\item programska potpora za platformu AWS, koja će ostvariti...,
\end{enumerate} 

\section{Programska potpora za mikrokontroler}

Ovdje ću ukratko navesti glavne značajke programske potpore za mikrokontroler.

Možda ovdje opisati korištene senzore kao i display? Ili u potpoglavlju? Definitivno važno za istaknuti!

\subsection{Arhitektura programske potpore za mikrokontroler}

Programska potpora za mikrokontroler temelji se na... 

Možda napraviti potpoglavlja u kojima opisujem korištene biblioteke na hardverskoj strani?

\section{Programska potpora za oblak}

Ovdje će trebati dosta toga navesti, odnosno koji su koraci napravljeni u samom AWS sustavu da bi se mikrokontroler uspješno spojio s AWS-om. Vjerojatno opisati korake kreiranja računa, stvaranja certifikata koji će se zatim posijati u uređaj itd. 

\subsection{Povezivanje s mikrokontrolerom}

Ovdje će trebati opisati cijeli flow kojim se mikrokontroler spaja na AWS.